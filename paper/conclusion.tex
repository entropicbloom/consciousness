We have established three main contributions:

\textbf{Theoretical framework.} We formalized the constraint that conscious representations must be unambiguous, emerging from consciousness being an intrinsic property. We defined ambiguity as conditional entropy $H(I|R)$ and proposed that relational structure can reduce this entropy by constraining possible interpretations. This framework does not require commitment to representationalism specifically; the same constraint follows from the intrinsicality and informativeness axioms of theories like IIT.

\textbf{Experimental operationalization.} We demonstrated that representational ambiguity can be measured empirically through decoding tasks. Networks trained with dropout achieve perfect (100\%) accuracy in class identity decoding and high accuracy (up to $R^2 = 0.844$) in spatial position decoding, corresponding to Ambiguity Reduction Scores approaching 1.0. The methodology extends naturally to other neural systems and representational domains.

\textbf{Practical insights.} Training paradigm dramatically affects representational ambiguity largely independently of task performance, at least within our tested regime. Dataset identity can be inferred from relational structure, and decoders generalize across different network architectures, suggesting that context dependence reflects practical rather than fundamental limitations. The consistency of relational geometries across network instances points toward the possibility of universal decoders that approach true $H(I|R)$.

These results establish that neural networks can achieve the unambiguous representations that theoretical accounts of consciousness require. While we make no claims about MNIST networks being conscious, our framework provides a quantitative method for assessing whether a neural substrate satisfies the intentionality constraint, a necessary (though likely not sufficient) condition for phenomenal experience. Future work might apply these methods to biological neural recordings to test whether low representational ambiguity is indeed associated with conscious processing, as the theoretical framework predicts.
