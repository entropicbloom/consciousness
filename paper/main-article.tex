% Standard Article Class Version
% For easy compilation without MDPI template
\documentclass[11pt,a4paper]{article}

% LaTeX packages
\usepackage[utf8]{inputenc}
\usepackage[T1]{fontenc}
\usepackage{geometry}
\geometry{margin=1in}
\usepackage{graphicx}
\usepackage{amsmath}
\usepackage{amssymb}
\usepackage{amsthm}
\usepackage{booktabs}
\usepackage{hyperref}
\usepackage{cleveref}
\usepackage{algorithm}
\usepackage{algpseudocode}
\usepackage{subcaption}
\usepackage{natbib}
\usepackage{times}

% Theorem environments
\newtheorem{definition}{Definition}
\newtheorem{corollary}{Corollary}
\newtheorem{theorem}{Theorem}
\newtheorem{proposition}{Proposition}

% Title and authors
\title{\textbf{Unambiguous Representations in Neural Networks: A Relational Structure Approach to Consciousness}}

\author{
    Anonymous Author$^{1}$\\
    \small $^{1}$Affiliation\\
    \small \texttt{anonymous@example.com}
}

\date{}

\begin{document}

\maketitle

\begin{abstract}
The neural correlates of consciousness (NCC) must explain not only that consciousness arises from neural activity, but also why specific patterns of neural activity correspond to specific conscious experiences. We argue that this intentionality constraint requires conscious representations to be unambiguous: they must intrinsically specify their content without relying on an arbitrary decoding scheme. Drawing on information theory, we formalize representational ambiguity as $H(I|R)$, the conditional entropy of possible interpretations given a representation. We propose that relational structures—where meaning emerges from patterns of relationships between elements—can achieve the required unambiguity. To test this hypothesis empirically, we train neural networks on image classification tasks and demonstrate that class identity and spatial position can be decoded from network connectivity using only relational information, achieving up to 100\% accuracy through geometric structure matching. We introduce the Ambiguity-Reduction Score (ARS) to quantify how much training reduces representational ambiguity, finding that dropout regularization produces representations with near-zero ambiguity ($\text{ARS} = 1.0$) for categorical content. Our results suggest that neural networks naturally develop unambiguous relational representations, providing a measurable framework for investigating the neural basis of conscious content.
\end{abstract}

\textbf{Keywords:} consciousness; neural correlates; representation; ambiguity; information theory; neural networks; relational structure

\section*{} % spacing

% Include sections
\section{Introduction}

Understanding the neural basis of consciousness remains one of the most challenging problems in science. The search for neural correlates of consciousness (NCCs)~\cite{koch2016neural} seeks to identify which patterns of neural activity give rise to specific conscious experiences. However, identifying correlations between brain states and conscious states is not sufficient—we must also explain \textit{why} a particular pattern of neural activity corresponds to one specific experience rather than another. This is fundamentally a question about representation and meaning: how do physical patterns in the brain come to be \textit{about} something?

Consider a simple analogy: a sequence of bits encoding a JPEG image. The same bit string could, in principle, be decoded as an image, a sound file, a text document, or any other digital format. The meaning of the bit string is entirely determined by the decoding algorithm applied to it—the representation itself is \textit{ambiguous}. This observation raises a crucial question for theories of consciousness: if conscious experiences have determinate content (I see an apple, not an orange), how can this determinacy arise from neural representations that might be fundamentally ambiguous?

Traditional representational theories of consciousness~\cite{lycan2019representational} propose that conscious experiences correspond to representations in the brain, but they often leave unanswered the question of what makes these representations be \textit{about} specific contents. One might be tempted to invoke an internal decoder—some mechanism in the brain that interprets neural patterns and gives them meaning. However, this approach leads to an infinite regress: the decoder's outputs would themselves be representations requiring interpretation, recreating the homunculus fallacy~\cite{dennett1991consciousness}.

We argue that this problem can be resolved by requiring that conscious representations be \textit{unambiguous}: they must intrinsically specify both what they represent and how they represent it, without relying on an arbitrary external decoding scheme. Drawing on information theory, we formalize this intuition by defining representational ambiguity as $H(I|R)$—the conditional entropy of possible interpretations $I$ given a representation $R$. Conscious representations, we propose, must approach zero ambiguity: $H(I|R) \approx 0$.

But how can any physical system achieve such unambiguous representations? We propose that \textit{relational structures}—where the meaning of elements is determined by their relationships to other elements—provide a solution. Unlike indexical representations (such as bit strings), relational structures embed meaning in their connectivity patterns. A neural network that has learned to model the statistical structure of its input domain naturally creates such relational representations: each element's meaning is determined by its position within the larger network of relationships that mirror real-world structure.

To test this theoretical framework empirically, we conduct experiments with artificial neural networks trained on image classification tasks. We demonstrate that:

\begin{enumerate}
    \item \textbf{Class identity can be decoded from relational structure alone}: Given the output layer weights of a network trained on MNIST, we can identify which digit each output neuron represents with up to 100\% accuracy using only the relational geometry between neurons, even when neuron labels are randomly permuted.

    \item \textbf{Spatial position information is encoded relationally}: We can decode the spatial position of input pixels purely from relational patterns in network connectivity, demonstrating that even low-level sensory structure emerges as unambiguous relational representations.

    \item \textbf{Training reduces representational ambiguity}: We introduce the Ambiguity-Reduction Score (ARS), which quantifies how much the training process reduces $H(I|R)$. Networks trained with dropout regularization achieve near-perfect unambiguity (ARS = 1.0) for categorical representations.

    \item \textbf{Relational geometry is architecture-invariant}: The same relational patterns emerge across different network architectures, suggesting these structures reflect fundamental properties of the learned distribution rather than architectural accidents.
\end{enumerate}

Our work makes several contributions. Theoretically, we provide a formal framework for understanding the intentionality constraint on neural correlates of consciousness, grounded in information theory. Empirically, we demonstrate that neural networks naturally develop unambiguous relational representations, and we provide methods to measure and quantify representational ambiguity. Methodologically, we show how geometric structure matching can decode representational content with near-perfect accuracy, suggesting practical approaches for reading out the contents of neural representations.

These results have implications beyond consciousness research. They connect to recent work on representational alignment in large-scale models~\cite{huh2024platonic}, suggest new approaches to interpretability in artificial neural networks, and provide a quantitative framework for comparing representations across different systems. Most fundamentally, they demonstrate that the problem of representational ambiguity—long considered a philosophical puzzle—can be approached through rigorous experimental investigation.

The remainder of this paper is organized as follows. Section~\ref{sec:theory} develops our theoretical framework, introducing the intentionality constraint and formalizing representational ambiguity. Section~\ref{sec:methods} describes our experimental methodology. Sections~\ref{sec:exp1} and~\ref{sec:exp2} present our experiments on digit classification and spatial position decoding, respectively. Section~\ref{sec:ambiguity} introduces the Ambiguity-Reduction Score and applies it to quantify our results. Section~\ref{sec:discussion} discusses implications, limitations, and connections to related work. Section~\ref{sec:conclusion} concludes.

\section{Theoretical Framework}
\label{sec:theory}

\subsection{Narrow Representationalism}

We can formalize the philosophical assumption underlying our work as \textit{narrow representationalism}. Representationalism holds that conscious systems instantiate representational properties—they represent objects or features of the world, and these intentional contents determine conscious experience. The ``narrow'' qualifier means these representational properties are completely determined by internal brain states, independent of external factors.

Under narrow representationalism, two molecularly identical brains would instantiate the same intentional contents and thus have the same subjective experience. If a brain state represents a red apple, that same physical state cannot alternatively represent a green square—the mapping from neural state to conscious content is \textit{unambiguous}. This stands in stark contrast to bit strings, where the same representation can mean different things depending on the decoder applied.

\subsection{The Intentionality Constraint}

This leads to a fundamental requirement for theories of consciousness:

\begin{definition}[Intentionality Constraint]
\label{def:intentionality}
The intentionality constraint on neural correlates of consciousness (NCCs) requires that an explanatory NCC must unambiguously represent the aspect of conscious experience it accounts for.
\end{definition}

\begin{corollary}
\label{cor:meaning}
Conscious representations must intrinsically specify both \textit{what} they represent and \textit{how} they represent it, without relying on an external decoder.
\end{corollary}

The intentionality constraint distinguishes conscious representations from arbitrary encodings like bit strings. When you see a red square, that experience is determined in the current moment—there is no need to compare your brain state against all possible experiences to give it meaning. Yet somehow, the meaning must be intrinsic to the representation itself.

\subsection{Formalizing Ambiguity}

We define ambiguity as the conditional entropy of possible interpretations:

\begin{equation}
\text{Ambiguity} = H(I|R)
\label{eq:ambiguity}
\end{equation}

where $H$ denotes entropy, $R$ is a representation, and $I$ represents the space of all possible interpretations. A representation with \textit{high ambiguity} has a uniform or near-uniform distribution over possible interpretations—we have little information about what it represents. A representation with \textit{low ambiguity} has a narrow distribution, concentrating probability on one or few interpretations.

For example, consider three possible interpretations $I = \{i_1, i_2, i_3\}$:
\begin{itemize}
    \item \textbf{Maximally ambiguous} (uniform): $p(i_1|R) = p(i_2|R) = p(i_3|R) = 1/3$ yields $H(I|R) = \log_2(3) \approx 1.58$ bits
    \item \textbf{Partially ambiguous}: $p(i_1|R) = 0.7, p(i_2|R) = 0.2, p(i_3|R) = 0.1$ yields $H(I|R) \approx 0.9$ bits
    \item \textbf{Unambiguous}: $p(i_1|R) = 1, p(i_2|R) = p(i_3|R) = 0$ yields $H(I|R) = 0$ bits
\end{itemize}

This gives us a spectrum of representations: bit strings are maximally ambiguous (could mean anything), while conscious brain states—given narrow representationalism—must be substantially less ambiguous, ideally approaching $H(I|R) = 0$.

\subsection{Context and Measurement}

In practice, we measure $H(I|R,C)$—the conditional entropy given both representation $R$ and context $C$. The context might include knowledge that we're dealing with neural networks trained on digit classification, or that interpretations should be digit class labels rather than arbitrary concepts. This differs from the theoretical ideal $H(I|R)$.

However, this limitation may be more practical than fundamental. Our experiments (Section~\ref{sec:exp1}) show that components of context—such as which dataset was used for training—can themselves be inferred from relational structure with near-perfect accuracy. This suggests that context and content are not sharply separable: what we treat as ``context'' may simply be aspects of $I$ we choose not to decode in a particular experiment.

For biological consciousness, the relevant context is presumably broad—``our part of the universe and our sensory modalities''—but still finite. Our framework suggests that within appropriately broad contexts, representations can achieve the required unambiguity.

\subsection{Relational Structures}

How can representations achieve unambiguity without external decoders? The key insight is that \textit{relational structures}—where elements derive meaning from their relationships to other elements—can specify content intrinsically.

Consider a simple example: encoding a 5×5 image of a square as a 25-dimensional vector (Figure~\ref{fig:pixel-encoding}a). While this \textit{looks} like a square to us, that's only because we've arranged the vector elements in a grid. The vector itself doesn't specify how it should be decoded—it's just 25 numbers. However, if we add relational information linking neighboring elements (Figure~\ref{fig:pixel-encoding}b), the representation becomes less arbitrary. Each element acquires meaning through its position in the relational structure.

\begin{figure}[h]
\centering
\includegraphics[width=0.8\textwidth]{../figures/pixel-encoding-relational-structure.png}
\caption{Pixel encoding and relational structure. \textbf{(a)} A binary vector visualized as a square grid. Without relational information, there's no intrinsic reason to arrange it this way. \textbf{(b)} Adding relations between neighboring elements constrains interpretation—the grid structure is now encoded in the representation itself.}
\label{fig:pixel-encoding}
\end{figure}

This idea has precedent in linguistics and consciousness research. Consider a dictionary: one might argue it's ambiguous because you need to know the language. However, from a structuralist perspective, the network of relationships between words—how they define each other—actually constrains meaning even without prior linguistic knowledge. A word derives partial meaning from occupying a specific position in the web of definitional relationships~\cite{lyre2022neurophenomenal}.

Relational structures as correlates of consciousness have been explored in depth by Kleiner and Ludwig~\cite{kleiner2024mathematical}, who argue that mathematical structures of consciousness must be intrinsically \textit{about} consciousness. Our approach complements theirs by framing the motivation in terms of representational ambiguity and information theory.

\subsection{Structuralism: Momentary vs. Potential}

An important distinction exists in how relational structures might ground meaning. Structuralism in philosophy of mind often emphasizes that experiential content is determined by relations to all \textit{potential} experiences a subject could have~\cite{lyre2022neurophenomenal}. However, for conscious experience to be determined \textit{in the current moment}, we need relational structure that is \textit{instantiated now}, not merely potential.

Our proposal focuses on this second type: conscious content is determined by the relational structure present in the current brain state. This structure may reflect or encode the space of possible experiences (learned through evolution and development), but it must be physically instantiated to determine current content.

\subsection{Physical Grounding}

For relational structures to avoid ambiguity at the physical level, the mapping from substrate to structure cannot be arbitrary. We need unambiguous correspondence between physical entities and structural elements. Key requirements include:

\begin{itemize}
    \item \textbf{Consistent physical grounding}: Elements and relations must correspond to identifiable physical entities or properties in a temporally stable way
    \item \textbf{Physical meaningfulness}: Relations between structural elements should be grounded in physical quantities that actually connect the corresponding physical entities
\end{itemize}

Ambiguity can arise at two levels: (1) the abstract mathematical structure itself (like a bit string), or (2) the physical implementation of an otherwise meaningful structure (like encoding a graph in arbitrary binary switches). Both must be addressed for truly unambiguous representation.

\subsection{Neural Networks and Learned Relational Structure}

How do neural networks relate to this framework? Consider that given a full distribution of natural images, we could potentially deduce the 2D grid structure from correlations between pixel values, even without knowing the encoding scheme a priori. A single image vector gains meaning from being embedded in the statistical structure of the distribution it's sampled from.

For conscious experience, meaning is determined in the current moment—we don't need to explicitly compare against all possible experiences. However, the brain may implicitly use distributional information learned through evolution and plasticity. By modeling the statistical structure of the natural world, neural networks embed each representation within a larger relational framework that reflects regularities in the environment.

This suggests neural networks themselves (not just their activation patterns) might constitute mathematical structures of unambiguous representation. By mirroring real-world relational structure, they avoid ambiguity at both abstract and physical levels:
\begin{itemize}
    \item \textbf{Abstract level}: The network encodes an intricate web of relations reflecting learned distributions
    \item \textbf{Physical level}: The network corresponds to actual physical nodes (neurons) and connections (synapses)
\end{itemize}

\subsection{Structural vs. Functional Connectivity}

Neural networks can instantiate relational structures through either:
\begin{enumerate}
    \item \textbf{Structural connectivity}: Presence and strength of synapses
    \item \textbf{Functional connectivity}: Statistical relationships between neural activities
\end{enumerate}

Theoretically, functional connectivity may be more appropriate for grounding conscious representations, as it captures actual ongoing relationships rather than mere possibilities. However, structural connectivity constrains and enables functional patterns. Our experiments (Sections~\ref{sec:exp1}-\ref{sec:exp2}) analyze structural connectivity as a stable proxy for the relational structure that functional connectivity would instantiate during processing.

\subsection{From Theory to Experiment}

Our theoretical framework makes a testable prediction: if neural networks develop unambiguous relational representations, then representational content should be decodable from relational structure alone, even when neuron identities are scrambled.

The experimental logic is straightforward. Consider an MNIST classifier with output neurons for digits 0-9. If I scramble the neurons (randomly permute them), can you figure out which neuron represents which digit using only the network's connectivity patterns? If representations are unambiguous, the answer should be yes—each digit occupies a unique position in the relational geometry that's consistent across different network instances learning the same distribution.

We test this by:
\begin{enumerate}
    \item Randomly permuting neurons to destroy positional information
    \item Computing relational structure (cosine similarities between neuron weights)
    \item Using decoders that see only relational information, never the same network twice
\end{enumerate}

Decoding accuracy directly relates to ambiguity:
\begin{itemize}
    \item Random performance (e.g., 10\% for 10 classes) indicates maximal ambiguity
    \item Above-chance performance indicates partial ambiguity reduction
    \item Perfect performance (100\%) indicates zero ambiguity: $H(I|R,C) = 0$
\end{itemize}

The methods section (Section~\ref{sec:methods}) provides full technical details of our experimental implementation.

\section{Methods}
\label{sec:methods}

\subsection{Overview}

To empirically test whether neural networks develop unambiguous relational representations, we designed experiments around a core principle: if representations are truly unambiguous, their content should be decodable purely from relational structure, even when indexical information (such as neuron positions or labels) is removed. Our approach involves three main components:

\begin{enumerate}
    \item \textbf{Base networks:} We train feedforward neural networks on image classification tasks (MNIST and Fashion-MNIST) using different training paradigms to generate networks with varying degrees of relational structure.
    \item \textbf{Dataset construction:} From each trained network, we extract weight matrices and construct datasets where neuron identities are deliberately obscured through random permutation, forcing any decoder to rely solely on relational information.
    \item \textbf{Decoders:} We employ two complementary approaches to decode representational content: (a) learned decoders based on self-attention architectures, and (b) geometric structure matching based on direct Gram matrix comparison.
\end{enumerate}

\subsection{Base Network Architecture and Training}

\subsubsection{Network Architecture}

We used fully-connected feedforward networks with the following specifications:
\begin{itemize}
    \item \textbf{Input layer:} 784 neurons (28×28 pixels)
    \item \textbf{Hidden layers:} Two hidden layers with 50 neurons each
    \item \textbf{Output layer:} 10 neurons (one per class)
    \item \textbf{Activation:} ReLU for hidden layers, softmax for output layer
\end{itemize}

For architectural invariance experiments, we varied the hidden layer sizes by randomly sampling widths between 25 and 100 neurons, or using fixed architectures such as [100] (single hidden layer with 100 neurons).

\subsubsection{Training Paradigms}

To investigate how different training procedures affect the emergence of relational structure, we employed three training paradigms:

\begin{enumerate}
    \item \textbf{Untrained (control):} Networks with randomly initialized weights, never trained. These serve as a negative control, as random weights should not contain meaningful relational structure.

    \item \textbf{Standard backpropagation:} Networks trained with standard stochastic gradient descent and cross-entropy loss for 2 epochs.

    \item \textbf{Backpropagation with dropout:} Networks trained with dropout regularization (rate = 0.2) applied to hidden layers during training. Dropout encourages distributed representations by randomly deactivating neurons during training~\cite{baldi2013understanding}.
\end{enumerate}

For all training paradigms, we used:
\begin{itemize}
    \item Optimizer: Adam
    \item Learning rate: 0.001
    \item Batch size: 256
    \item Epochs: 2 (0 for untrained paradigm)
\end{itemize}

To generate sufficient data for decoder training and evaluation, we trained 1000 instances of each base network type using different random seeds.

\subsection{Dataset Construction for Relational Decoding}

\subsubsection{Output Neuron Class Identity Dataset}

For Experiment 1 (Section~\ref{sec:exp1}), we construct a dataset to test whether output neuron class identity can be decoded from relational structure. Let $\mathbf{W} \in \mathbb{R}^{H \times 10}$ denote the output layer weight matrix of a trained network, where $H$ is the dimensionality of the last hidden layer and the 10 columns correspond to the 10 output neurons (one per digit class 0-9).

To create a single training example:
\begin{enumerate}
    \item Apply a random permutation $\pi$ to the columns of $\mathbf{W}$, yielding $\mathbf{X} = \mathbf{W}\pi$.
    \item The target $y$ is the class index of the output neuron that ended up in the first column after permutation.
\end{enumerate}

Each of the 1000 trained networks contributes 10 data points (one per output neuron), yielding 10,000 total examples. We split these into 8,000 training and 2,000 validation examples, ensuring that all data points from the same network instance appear in only one split.

\subsubsection{Input Neuron Spatial Position Dataset}

For Experiment 2 (Section~\ref{sec:exp2}), we construct a dataset to test whether input neuron spatial position can be decoded from relational structure. Let $\mathbf{W} \in \mathbb{R}^{784 \times H}$ denote the first layer weight matrix (transposed view), where the 784 rows correspond to input pixels and $H$ is the first hidden layer size.

To create a single training example:
\begin{enumerate}
    \item Apply a random permutation $\pi$ to the rows of $\mathbf{W}$, yielding $\mathbf{X} = \pi \mathbf{W}$.
    \item Define a target function $f(i,j)$ that extracts positional information from pixel coordinates $(i,j)$, such as:
    \begin{itemize}
        \item Distance from center: $f(i,j) = \sqrt{(i-13.5)^2 + (j-13.5)^2}$
        \item Vertical position: $f(i,j) = j/27$
        \item Horizontal position: $f(i,j) = i/27$
    \end{itemize}
    \item The target $y = f(i,j)$ for the input neuron that ended up in the first row.
\end{enumerate}

\subsection{Relational Preprocessing}

To make the decoding task explicitly relational, we transform weight matrices into cosine similarity matrices that encode pairwise relationships between neurons.

For output neurons (Experiment 1), we compute:
\begin{align}
\mathbf{X}_{norm} &= \frac{\mathbf{X}}{\|\mathbf{X}\|_{col}} \\
\mathbf{X}' &= \mathbf{X}_{norm}^T \mathbf{X}_{norm}
\end{align}

where $\|\mathbf{X}\|_{col}$ denotes column-wise L2 normalization. Each element $(\mathbf{X}')_{i,j}$ represents the cosine similarity between the incoming weights of output neurons $i$ and $j$.

For input neurons (Experiment 2), we compute:
\begin{align}
\mathbf{X}_{norm} &= \frac{\mathbf{X}}{\|\mathbf{X}\|_{row}} \\
\mathbf{X}' &= \mathbf{X}_{norm} \mathbf{X}_{norm}^T
\end{align}

where each element $(\mathbf{X}')_{i,j}$ represents the cosine similarity between the outgoing weights of input neurons $i$ and $j$.

This preprocessing serves two purposes: (1) it emphasizes relational structure by encoding similarities rather than raw weights, and (2) it removes information about weight magnitudes that could provide trivial shortcuts for decoding.

\subsection{Decoder Architectures}

\subsubsection{Self-Attention Decoder}

Our primary learned decoder uses a Set Transformer architecture~\cite{vaswani2017attention} that is inherently permutation-invariant. The architecture consists of:

\begin{enumerate}
    \item \textbf{Input:} The similarity matrix $\mathbf{X}' \in \mathbb{R}^{N \times N}$, where $N$ is the number of neurons (10 for output neurons, 784 for input neurons). Rows are treated as tokens.

    \item \textbf{Embedding:} Each row (token) is linearly embedded to a hidden dimension of 128.

    \item \textbf{Multi-head self-attention layers:} Two layers of multi-head self-attention (4 heads each) process the embedded tokens. These layers compute relationships between all tokens simultaneously while maintaining permutation invariance.

    \item \textbf{Read-out:} After processing through attention layers, we extract only the representation corresponding to the first token (the target neuron).

    \item \textbf{Output layer:} A linear layer maps the extracted representation to the output space:
    \begin{itemize}
        \item For classification (Experiment 1): 10-dimensional output with cross-entropy loss
        \item For regression (Experiment 2): 1-dimensional output with mean squared error loss
    \end{itemize}
\end{enumerate}

The decoder is trained for 200 epochs using:
\begin{itemize}
    \item Optimizer: Adam
    \item Learning rate: 0.001
    \item Batch size: 64
\end{itemize}

Critically, the decoder never sees the same base network instance in both training and validation, ensuring it learns general relational principles rather than memorizing network-specific patterns.

\subsubsection{Geometric Structure Matching}

As an alternative to learned decoders, we developed a geometric matching approach that directly compares relational structures without requiring training. This method constructs a reference geometry by averaging similarity matrices from several reference networks, then decodes test networks by finding the permutation that best aligns their geometry to the reference.

Formally, given $K$ reference networks with similarity matrices $\{\mathbf{G}^{(k)}\}_{k=1}^K$, we compute:
\begin{equation}
\mathbf{G}_{ref} = \frac{1}{K} \sum_{k=1}^K \mathbf{G}^{(k)}
\end{equation}

For a test network with permuted neurons yielding similarity matrix $\mathbf{G}_{test}$, we evaluate all possible permutations $\pi$ and select:
\begin{equation}
\pi^* = \argmin_{\pi} \|\mathbf{G}_{ref} - \pi \mathbf{G}_{test} \pi^T\|_F
\end{equation}

where $\|\cdot\|_F$ denotes the Frobenius norm. The permutation $\pi^*$ that minimizes the distance to the reference geometry provides our decoding.

For output neurons with 10 classes, this requires evaluating $10! = 3,628,800$ permutations. For larger sets, we use approximation algorithms or restrict to subsets.

\subsection{Evaluation Metrics}

\begin{itemize}
    \item \textbf{Classification accuracy:} For Experiment 1, we report top-1 accuracy in identifying the correct class of the target output neuron.
    \item \textbf{$R^2$ score:} For Experiment 2, we report the coefficient of determination for predicting continuous spatial properties.
    \item \textbf{Ambiguity-Reduction Score (ARS):} A metric derived from accuracy or $R^2$ that quantifies the reduction in representational ambiguity (see Section~\ref{sec:ambiguity}).
\end{itemize}

All results report mean and standard deviation across 5 random seeds for decoder training.

\subsection{Ablation Studies}

To validate that decoders exploit full relational structure rather than local patterns, we conduct two types of ablation:

\begin{enumerate}
    \item \textbf{Target similarity only:} Provide the decoder with only the first row of $\mathbf{X}'$ (the target neuron's similarities to all others), masking out all pairwise similarities between non-target neurons.

    \item \textbf{Subset size variation:} Randomly sample subsets of neurons of varying sizes and perform decoding on the reduced relational structure.
\end{enumerate}

\subsection{Cross-Architecture Transfer}

To test whether relational structure is architecture-invariant, we train decoders on networks with one hidden layer configuration and evaluate on networks with different configurations. We also test the geometric matching approach across architectures by using reference networks of one architecture to decode test networks of another.

\section{Experiment 1: Decoding MNIST Digit Class from Relational Structure}
\label{sec:exp1}

\subsection{Setup}

Imagine an MNIST classifier where each output neuron corresponds to a digit (0-9). Normally, if you know how the network was trained, determining what a neuron represents is trivial—the neuron at position 2 represents the digit 2 because that's how we set it up.

But now suppose I scramble all the neurons in each layer. You no longer know which neuron position corresponds to which digit. The question is: \textit{can you figure out what each neuron represents purely from the network's connectivity structure?}

Our hypothesis: if the network unambiguously represents digit concepts through relational structure, then yes—each digit should occupy a unique position in the relational geometry that's recoverable even after scrambling. We test this by extracting relational structure (cosine similarities between incoming weights) and attempting to decode class identity from this structure alone.

\subsection{Results}

We trained decoders on three types of networks, varying only in how the underlying MNIST classifiers were trained:

\begin{enumerate}
    \item \textbf{Untrained (control)}: Random weights, never trained
    \item \textbf{Standard backpropagation}: Standard training with SGD
    \item \textbf{Backpropagation with dropout}: Training with 20\% dropout regularization
\end{enumerate}

Figure~\ref{fig:decoder-accuracy} shows decoder validation accuracy during training.

\begin{figure}[h]
\centering
\includegraphics[width=0.8\textwidth]{../figures/decoder-validation-accuracy-training-paradigms.png}
\caption{Decoder validation accuracy across training paradigms. Self-attention decoder learns to identify output neuron classes from relational structure. Three training paradigms refer to how the underlying MNIST networks were trained. Error margins show standard deviation across 5 decoder training seeds.}
\label{fig:decoder-accuracy}
\end{figure}

The results are striking:

\begin{itemize}
    \item \textbf{Untrained}: $\sim$10\% accuracy (chance level for 10 classes)—random weights have no meaningful structure
    \item \textbf{Standard backpropagation}: $\sim$25\% accuracy—well above chance, but substantial ambiguity remains
    \item \textbf{Dropout}: $\sim$75\% accuracy—dramatic improvement, showing highly distinctive relational structure
\end{itemize}

Dropout encourages distributed representations by forcing neurons to rely on population activity rather than individual pathways~\cite{baldi2013understanding}. Output neurons representing similar digits develop similar weight patterns because they must share overlapping features, creating distinctive relational geometries.

\subsection{Task Performance vs. Representational Ambiguity}

An important observation: dropout's dramatic effect on decoding accuracy ($\sim$75\% vs. $\sim$25\%) occurs despite both training methods achieving virtually identical MNIST classification performance.

\begin{figure}[h]
\centering
\includegraphics[width=0.6\textwidth]{../figures/mnist-model-validation-accuracies.png}
\caption{MNIST classification accuracies of the underlying networks. Both dropout and standard backpropagation achieve $\sim$97\% accuracy on the digit classification task itself, despite vastly different representational ambiguity.}
\label{fig:mnist-accuracy}
\end{figure}

Figure~\ref{fig:mnist-accuracy} shows that both training paradigms achieve $\approx 97\%$ MNIST accuracy. Yet their internal representations differ dramatically in ambiguity (25\% vs. 75\% decoding accuracy). This dissociation demonstrates that \textit{representational ambiguity is largely independent of task performance}—networks can solve the same task equally well while organizing information very differently. Dropout fundamentally changes \textit{how} information is represented, creating more distinctive relational geometries, without changing \textit{whether} the task is solved.

\subsection{Ablation: Importance of Full Relational Structure}

To verify that the decoder exploits the full relational geometry rather than just local patterns around the target neuron, we reran the experiment providing only the first row of $\mathbf{X}'$ (the target neuron's cosine similarities to all others), while masking out all pairwise similarities between non-target neurons.

\begin{figure}[h]
\centering
\includegraphics[width=0.7\textwidth]{../figures/target-similarity-only-output-neurons.png}
\caption{Ablation: target similarity only for output neurons. Decoder accuracy when provided with full relational structure (blue) versus only the target neuron's local neighborhood (orange). Accuracy plummets when contextual structure is removed, demonstrating that the decoder requires the full relational geometry to disambiguate class identity.}
\label{fig:ablation-target-only}
\end{figure}

Figure~\ref{fig:ablation-target-only} shows that accuracy drops substantially when contextual structure is removed. A single neuron's local neighborhood is insufficient to accurately determine its class identity; the decoder takes into account how the rest of the output population is organized to disambiguate which digit the target neuron represents. This confirms that unambiguous representation emerges from the full relational structure, not merely from local pairwise relationships.

\subsection{Ablation: Effect of Neuron Count}

To investigate how relational structure complexity affects decoding accuracy, we systematically reduced the number of output neurons available to the decoder. Figure~\ref{fig:ablation-neuron-count} shows both absolute accuracy and performance relative to random guessing across different neuron counts.

\begin{figure}[h]
\centering
\includegraphics[width=0.8\textwidth]{../figures/ablation-study-neuron-count-performance.png}
\caption{Ablation: neuron count performance. Top panel compares validation accuracy (purple) to random guessing baseline (green) for networks with 2-10 output neurons. Bottom panel shows relative performance gain compared to random guessing. The 2-neuron case achieves only random-level performance (1.0×), validating that asymmetric relational structures are necessary. Performance relative to random chance increases consistently with neuron count, with the 10-neuron model achieving 7.36× better than random.}
\label{fig:ablation-neuron-count}
\end{figure}

Our results confirm that asymmetric relational structures between output neurons are essential for decoding to function. The 2-neuron case performs exactly at random chance level (50\%, or 1.0×), since asymmetric relations cannot exist between only two points. While the 5-neuron condition achieves the highest absolute validation accuracy (79.1\%), performance relative to random guessing increases consistently with neuron count. The 10-neuron model performs 7.36× better than random (73.6\% absolute), suggesting that decoder accuracy should continue to improve with more complex input distributions as the ambiguity of underlying representations decreases.

\subsection{Geometric Matching: Perfect Decoding}

Having established that learned decoders can extract relational structure, we tested a more direct approach: geometric matching. Instead of training a decoder, we construct a reference geometry from 5 networks and find which permutation of a test network's neurons best aligns with this reference (by evaluating all 10! = 3,628,800 possible permutations using Frobenius distance).

Table~\ref{tab:gram-accuracy} shows the results:

\begin{table}[h]
\centering
\caption{Gram matrix decoding accuracies using 5 reference networks and 10 validation networks.}
\label{tab:gram-accuracy}
\begin{tabular}{lcc}
\toprule
\textbf{Training Paradigm} & \textbf{Accuracy} & \textbf{Std Dev} \\
\midrule
Untrained       & 0.100 & 0.155 \\
No dropout      & 0.383 & 0.441 \\
Dropout         & \textbf{1.000} & \textbf{0.000} \\
\bottomrule
\end{tabular}
\end{table}

The dropout condition achieves \textbf{perfect 100\% accuracy with zero variance}. Every single test network's true class ordering minimizes the distance to the reference geometry, with no exceptions. This means the relational structure is not just similar across networks—it is effectively identical up to permutation. The representation is \textit{completely unambiguous}: $H(I|R,C) = 0$ for digit class identity.

Geometric matching achieves higher accuracies than the learned decoder (100\% vs. 75\%) while using far fewer networks (5 references vs. 800 for training). Untrained networks perform at chance (10\%) as expected, while standard backpropagation achieves 38.3\%—much higher than the learned decoder's 25\%, but far below dropout's perfect score.

\subsubsection{Understanding Perfect Accuracy: Permutation Distance Distributions}

To understand why dropout achieves perfect accuracy while vanilla backpropagation struggles, we examined the distribution of Frobenius distances for all possible permutations. Figures~\ref{fig:perm-dist-nodropout} and~\ref{fig:perm-dist-dropout} show these distributions for representative networks.

\begin{figure}[h]
\centering
\begin{subfigure}{0.48\textwidth}
\includegraphics[width=\textwidth]{../figures/perm_distances_no_dropout.png}
\caption{No dropout}
\label{fig:perm-dist-nodropout}
\end{subfigure}
\hfill
\begin{subfigure}{0.48\textwidth}
\includegraphics[width=\textwidth]{../figures/perm_distances_dropout.png}
\caption{Dropout}
\label{fig:perm-dist-dropout}
\end{subfigure}
\caption{Permutation distance distributions. Histograms show Frobenius distances between reference and test Gram matrices across all possible permutations. Red dots indicate the true permutation. \textbf{(a)} For networks without dropout, the true permutation has only a small margin over incorrect permutations, making it easily confused with alternatives. \textbf{(b)} For dropout networks, the true permutation shows a substantial gap from all incorrect alternatives, creating an unambiguous geometric signature.}
\label{fig:perm-dist}
\end{figure}

For networks without dropout (Figure~\ref{fig:perm-dist-nodropout}), the true permutation (red dot) has only a tiny margin over incorrect permutations, making it easily confused with alternatives. In contrast, dropout networks (Figure~\ref{fig:perm-dist-dropout}) show a substantial gap between the correct permutation and all others, creating an unambiguous geometric signature that reliably identifies the true class ordering.

This geometric clarity explains why both our learned decoder approach (75\% accuracy) and direct matching approach (100\% accuracy) achieve their best performance on dropout networks—the underlying relational structure is fundamentally more distinctive and consistent.

\subsubsection{Interpreting Perfect Accuracy: Complete Unambiguity}

The 100\% accuracy achieved by geometric matching on dropout networks deserves careful interpretation, as it represents a remarkable result both empirically and theoretically.

\textbf{What 100\% accuracy means:} Perfect decoding accuracy indicates that, across all tested network instances, the true class ordering minimizes the distance to the reference geometry with zero exceptions. This means that the relational geometry of the output layer is not merely similar across networks—it is effectively identical up to permutation. Each digit class occupies a consistent, unique position in the 10-dimensional relational space that is perfectly preserved across different random initializations and training trajectories.

\textbf{Theoretical implications:} In terms of our ambiguity framework, perfect accuracy implies $H(I|R,C) = 0$ for categorical class identity within the MNIST context. Given the relational structure of a dropout-trained network, there is zero remaining entropy about which neuron represents which digit. The representation is \textit{completely unambiguous} with respect to this interpretive task.

This is precisely what the intentionality constraint (Definition~\ref{def:intentionality}) requires: the representation intrinsically specifies its content without any remaining ambiguity. Unlike a bit string, which could mean anything depending on the decoder applied, the relational geometry unambiguously specifies ``this neuron represents the digit 3'' and no other interpretation is consistent with the structural evidence.

\textbf{Why dropout enables this:} The perfect consistency suggests that dropout fundamentally changes how networks represent categorical information. Rather than allowing each network to find an arbitrary solution within the space of functionally equivalent representations, dropout constrains learning to converge on a canonical relational structure. This structure reflects the intrinsic geometry of the MNIST distribution itself—the ways in which different digits are objectively similar or dissimilar in visual feature space.

\textbf{Contrast with standard training:} The 38\% accuracy for standard backpropagation (vs. 100\% for dropout) reveals that task performance alone is insufficient to guarantee unambiguous representations. Both training paradigms achieve $\sim$97\% MNIST classification accuracy, yet differ dramatically in representational ambiguity. Standard training finds solutions that work for the task but retain substantial ambiguity in their relational structure. Dropout finds solutions that not only work but do so through unambiguous encoding.

\textbf{Broader significance:} This result demonstrates that neural networks \textit{can} achieve the degree of unambiguity that consciousness theories require, at least for categorical representations in constrained domains. Whether biological neural networks achieve similar unambiguity, and whether this extends to richer phenomenal contents, remains to be tested. But the proof of principle is established: unambiguous relational representations are achievable in physical systems through learning.

\subsection{Cross-Architecture Transfer}

To test whether relational structure is architecture-invariant, we evaluated cross-architecture transfer for both learned decoders and geometric matching.

\subsubsection{Learned Decoder Transfer}

Figure~\ref{fig:arch-transfer-decoder} summarizes results for an unseen target architecture [100] (single hidden layer with 100 neurons).

\begin{figure}[h]
\centering
\includegraphics[width=0.6\textwidth]{../figures/architecture-transfer-evaluation.png}
\caption{Architecture transfer evaluation for learned decoder. A decoder trained only on [50, 50] architecture transfers above chance to [100] architecture. A decoder trained on networks with randomly sampled layer widths (25-100) achieves near-oracle performance on the held-out architecture, demonstrating architecture-independent generalization.}
\label{fig:arch-transfer-decoder}
\end{figure}

A decoder trained only on [50, 50] already transfers above chance, while a decoder trained on networks whose layer width is randomly sampled between 25 and 100 climbs almost to the self-transfer ``oracle,'' demonstrating near-architecture-independent generalization.

\subsubsection{Geometric Matching Transfer}

Figure~\ref{fig:arch-transfer-gram} shows cross-architecture transfer results for the Gram matrix matching approach.

\begin{figure}[h]
\centering
\includegraphics[width=0.7\textwidth]{../figures/cross_architecture_heatmap_accuracy.png}
\caption{Cross-architecture transfer heatmap for geometric matching. Each cell shows decoding accuracy when using reference networks of one architecture (y-axis) to decode test networks of another architecture (x-axis). Strong diagonal and near-diagonal performance demonstrates that the gram matrix approach maintains high accuracy across different network architectures.}
\label{fig:arch-transfer-gram}
\end{figure}

The heatmap reveals strong diagonal and near-diagonal performance, confirming that the gram matrix approach maintains high accuracy across different network architectures. This architecture-invariance suggests that the relational geometric structure reflects fundamental properties of the learned distribution rather than architectural accidents.

\subsection{Ablation: Neuron Count for Geometric Matching}

Similar to our ablation study with the learned decoder, we investigated how the number of output neurons affects the Gram matrix matching approach.

\begin{figure}[h]
\centering
\includegraphics[width=0.7\textwidth]{../figures/gram_neuron_ablation_plot.png}
\caption{Gram matrix decoding performance vs. number of neurons. Ablation study showing how Gram matrix matching accuracy varies with the number of output neurons available for decoding. Performance improves consistently as more relational structure becomes available.}
\label{fig:gram-neuron-ablation}
\end{figure}

Figure~\ref{fig:gram-neuron-ablation} shows that, consistent with the learned decoder results, accuracy improves as more output neurons (and thus more relational structure) become available. This reinforces the conclusion that richer relational structures enable more unambiguous representations.

\subsection{Dataset Classification from Relational Structure}

As a final test of how much information is encoded in relational structure, we trained a decoder to classify whether a network was trained on MNIST or Fashion-MNIST based solely on output layer weights with randomly permuted neurons.

\begin{figure}[h]
\centering
\includegraphics[width=0.6\textwidth]{../figures/dataset-classification-accuracy.png}
\caption{Dataset classification from final-layer weights. Using the same self-attention decoder with random output-neuron permutations, we classify whether a network was trained on MNIST or Fashion-MNIST. Performance is near-perfect for dropout models (0.998 ± 0.001) and clearly above chance for no dropout (0.843 ± 0.008).}
\label{fig:dataset-classification}
\end{figure}

Figure~\ref{fig:dataset-classification} shows near-perfect classification accuracy (0.998 ± 0.001) for dropout models and above-chance accuracy (0.843 ± 0.008) for standard backpropagation. This demonstrates that the relational geometry of output weights carries a dataset-specific signature, amplified by dropout. The fact that task domain can be inferred from relational structure suggests that key components of representational context are actually encoded within the representation itself.

\subsection{Can We Conclude Zero Ambiguity?}

The perfect 100\% decoding accuracy for dropout networks is remarkable—it indicates $H(I|R,C) = 0$ within our experimental context. However, can we conclude that these representations achieve truly zero ambiguity, or \textit{unconditional} $H(I|R) = 0$?

The answer is no, not yet. Our decoder has significant context baked in:

\begin{itemize}
    \item \textbf{Dataset}: The decoder knows networks were trained on MNIST, not other datasets
    \item \textbf{Architecture}: The decoder uses reference networks with known architectures
    \item \textbf{Task}: The decoder knows to predict digit class identity, not other properties
\end{itemize}

This context constrains the interpretation space. We measure $H(I|R,C)$ (entropy given representation \textit{and} context) rather than ideal $H(I|R)$ (entropy given representation alone).

However, two findings suggest this is a practical rather than fundamental limitation:

\subsubsection{Architecture Generalization}

Cross-architecture transfer (Figure~\ref{fig:arch-transfer-gram}) shows that relational structure generalizes across different network sizes. Using reference networks of one architecture to decode targets with different architectures achieves 75-100\% accuracy in most cases.

\begin{figure}[h]
\centering
\includegraphics[width=0.7\textwidth]{../figures/cross_architecture_heatmap_accuracy.png}
\caption{Cross-architecture transfer for geometric matching. High accuracy across different architectures (heatmap rows=reference, columns=target) shows relational structure transcends architectural specifics.}
\label{fig:arch-transfer-gram}
\end{figure}

This suggests network architecture is not a crucial component of the context—the relational structure reflects properties of the learned distribution more than architectural details.

\subsubsection{Dataset Discrimination}

Can we infer which dataset was used from relational structure alone? Figure~\ref{fig:dataset-classification} shows the answer is yes.

\begin{figure}[h]
\centering
\includegraphics[width=0.6\textwidth]{../figures/dataset-classification-accuracy.png}
\caption{Dataset classification from output layer weights. Using scrambled neurons, we classify whether networks were trained on MNIST vs. Fashion-MNIST with near-perfect accuracy (99.8\% for dropout, 84.3\% for standard training).}
\label{fig:dataset-classification}
\end{figure}

We achieve $\sim$100\% accuracy distinguishing MNIST from Fashion-MNIST networks based solely on relational structure. This shows that task domain—a major component of context $C$—is actually encoded in the representation $R$ itself. The distinction between ``content to decode'' ($I$) and ``context'' ($C$) is somewhat artificial: context is simply aspects of $I$ we choose not to decode in a particular experiment.

These results suggest that with richer, more diverse training and more sophisticated decoders, we could approach true $H(I|R)$ rather than just $H(I|R,C)$.

\section{Experiment 2: Decoding Spatial Position from Relational Structure}
\label{sec:exp2}

\subsection{Motivation}

Experiment 1 demonstrated that abstract categorical information (digit class identity) can be decoded from relational structure. However, because of the abstract nature of class identity, the connection to phenomenal consciousness might be unintuitive—especially if one thinks of phenomenal consciousness as applying more to sensory than to abstract representations.

To address this, Experiment 2 examines whether spatial structure—a fundamental aspect of sensory experience—is also encoded relationally. This connects directly to the theoretical example of the 2D pixel grid in Section~\ref{sec:theory}. We know that input neurons represent a grid of pixels in terms of how they are used during forward passes. But is there a sense in which this spatial information is intrinsic to the network connectivity itself?

We frame this as a decoding problem: Given the first layer weight matrix with permuted columns (where rows correspond to first hidden layer neurons and columns to input neurons), can we infer positional information about input neurons? While the strictest version would be to identify exact coordinates, we also consider weaker versions such as identifying a single coordinate or distance from center.

\subsection{Methodological Considerations}

Several design choices constrain our approach:

\begin{enumerate}
    \item \textbf{Single layer analysis:} We examine only the first layer to (a) simplify the experiment and reduce degrees of freedom, and (b) preclude solutions involving passing sample inputs through the full network to test how input neurons affect outputs for different inputs.

    \item \textbf{Relational preprocessing:} Using cosine similarity preprocessing prevents trivial solutions such as inferring position from weight magnitude norms.

    \item \textbf{Note on structural vs. functional connectivity:} As in Experiment 1, we analyze structural connectivity (synaptic weights) rather than functional connectivity (activity correlations). While functional connectivity may ultimately be the proper grounding for conscious representations, structural connectivity should reflect and enable functional connectivity patterns.
\end{enumerate}

\subsection{Preliminary Evidence}

Before developing our decoder, we visualized the relational structure using UMAP dimensionality reduction on the cosine similarity matrix between input neurons of a trained network (without dropout).

\begin{figure}[h]
\centering
\includegraphics[width=0.7\textwidth]{../figures/umap-input-neuron-similarity.png}
\caption{UMAP visualization of input neuron cosine similarity vectors. Two-dimensional embedding of the 784 input neurons based on their relational similarities, colored by spatial position in the 28×28 grid. Clear spatial clustering emerges, suggesting that positional information is present and decodable from relational structure alone.}
\label{fig:umap-input}
\end{figure}

Figure~\ref{fig:umap-input} shows clear spatial clustering, with input neurons organizing according to their position in the 28×28 grid. This provides preliminary evidence that positional information is indeed present in the relational structure and should be decodable.

\subsection{Task Formulation}

We cast spatial decoding as a supervised learning problem. We define a function $f(i,j)$ that extracts positional information from an input neuron's location $(i,j)$ in the 28×28 grid. The decoder's task is to predict $f(i,j)$ \textbf{given only the relational representation} of that neuron relative to all others—crucially, \textbf{without knowing} the values of $i$ or $j$ directly.

Example target functions include:
\begin{align}
f(i,j) &= \sqrt{(i - 13.5)^2 + (j - 13.5)^2} \quad \text{(distance from center)} \\
f(i,j) &= i/27 \quad \text{(normalized horizontal position)} \\
f(i,j) &= j/27 \quad \text{(normalized vertical position)}
\end{align}

For our main results, we focus on distance from center as it provides a single continuous target that captures 2D spatial structure.

\subsection{Dataset Construction}

As in Experiment 1, we generate multiple network instances with different initialization seeds. For each network, we extract the first layer weight matrix $\mathbf{W} \in \mathbb{R}^{784 \times H}$ (transposed view), where 784 corresponds to input pixels and $H$ is the first hidden layer size.

To build one training example:
\begin{enumerate}
    \item Permute the rows of $\mathbf{W}$, destroying positional information.
    \item Select one row (one input neuron) and place it first.
    \item The input $\mathbf{X}'$ to the decoder is the cosine similarity matrix computed from permuted weights (see Section~\ref{sec:methods}).
    \item The label $y = f(i,j)$ for the input neuron in the first position.
\end{enumerate}

Crucially, the decoder never sees coordinates $(i,j)$ directly—only connectivity patterns—and must infer positional information from relational structure alone.

\subsection{Results: Distance from Center Prediction}

Figure~\ref{fig:input-distance-accuracy} shows the $R^2$ scores for predicting distance from center across the three training paradigms.

\begin{figure}[h]
\centering
\includegraphics[width=0.8\textwidth]{../figures/input-neuron-distance-prediction-accuracy.png}
\caption{Input neuron distance prediction accuracy. $R^2$ scores for predicting pixel distance from center using relational structure alone, across three training paradigms. Error bars show standard deviation across 5 decoder training seeds.}
\label{fig:input-distance-accuracy}
\end{figure}

The results demonstrate that spatial position decoding is feasible:

\begin{itemize}
    \item \textbf{Untrained networks (control):} Achieve $R^2 \approx 0$, confirming that random weights contain no spatial information.

    \item \textbf{Standard backpropagation:} Achieve $R^2 \approx 0.84$, indicating substantial spatial structure in relational representations.

    \item \textbf{Backpropagation with dropout:} Achieve $R^2 \approx 0.70$, interestingly \textit{lower} than standard backpropagation.
\end{itemize}

The finding that dropout \textit{reduces} performance for spatial decoding contrasts sharply with Experiment 1, where dropout dramatically improved class identity decoding. This suggests a fundamental difference in how categorical vs. spatial information is encoded: dropout may enhance distinctiveness of categorical representations while making spatial representations more distributed and thus harder to decode from structural connectivity alone.

\subsection{Ablation: Importance of Full Relational Structure}

As in Experiment 1, we tested whether the decoder exploits full relational geometry or merely local neighborhoods by providing only the target neuron's cosine similarity row.

\begin{figure}[h]
\centering
\includegraphics[width=0.7\textwidth]{../figures/target-similarity-only-input-pixels.png}
\caption{Ablation: target similarity only for input pixels. Decoder performance when provided with full relational structure versus only the target neuron's local neighborhood. While the decoder can extract useful information from local neighborhoods alone, adding the full relational structure significantly improves performance.}
\label{fig:ablation-input}
\end{figure}

Figure~\ref{fig:ablation-input} shows that while the decoder can indeed extract useful information solely from the local neighborhood of the target neuron, adding more relations significantly improves performance. This validates our intuition that richer relational structure helps disambiguate representations—though the effect is less dramatic than in Experiment 1, suggesting that spatial information has more local character than categorical information.

\subsection{Effect of Relational Structure Size}

To assess how the size of the relational graph affects decoding accuracy, we sampled uniformly random subsets of the 784 input neurons and performed decoding on subsets of varying sizes.

\begin{figure}[h]
\centering
\includegraphics[width=0.7\textwidth]{../figures/varying-subset-size-input-pixels.png}
\caption{Effect of subset size on spatial decoding. Decoder $R^2$ score as a function of the number of input neurons included in the relational structure. Performance improves with larger subsets but shows diminishing returns, indicating that intentional content can be extracted even from subgraphs of the full relational structure.}
\label{fig:subset-size}
\end{figure}

Figure~\ref{fig:subset-size} reveals that while performance improves as we increase the size of the relational graph, adding more neurons yields diminishing returns beyond a certain point. This has important theoretical implications: intentional content can be extracted from neurons even when considering only a subgraph of the relational structure they are embedded in. This suggests that conscious representations might not require global integration of all neural activity, but rather a sufficiently rich local relational structure.

\subsection{Discussion}

The results of Experiment 2 demonstrate that spatial structure—a core component of sensory experience—is encoded in relational representations:

\begin{enumerate}
    \item \textbf{Spatial information is relationally encoded:} Achieving $R^2 \approx 0.84$ for position decoding shows that 2D spatial structure emerges from relational connectivity patterns, supporting the theoretical claim that relational structures can ground spatial aspects of phenomenal experience.

    \item \textbf{Different information types have different relational signatures:} The contrasting effects of dropout on categorical (improves) vs. spatial (degrades) decoding suggest that different aspects of experience may be encoded through different relational mechanisms.

    \item \textbf{Partial structures suffice:} The finding that subgraphs of relational structure can support decoding suggests that unambiguous representation does not require integrating all neural activity—a sufficiently rich local structure may suffice.

    \item \textbf{Structural connectivity reflects learned structure:} The fact that untrained networks show no spatial structure while trained networks do confirms that relational structure emerges through learning to model input distributions.
\end{enumerate}

Together with Experiment 1, these results demonstrate that neural networks develop unambiguous relational representations for both categorical and spatial information, providing empirical support for the theoretical framework developed in Section~\ref{sec:theory}.

\section{Quantifying Representational Ambiguity}
\label{sec:ambiguity}

\subsection{From Decoder Accuracy to Ambiguity}

In Section~\ref{sec:theory}, we defined representational ambiguity as $H(I|R)$: the entropy over all possible interpretations given a representation. Our experiments demonstrated that decoders can recover representational content by analyzing relational structure. However, to link decoder performance more directly to the theoretical concept of ambiguity, we need a quantitative framework.

In practice, we never possess a ``God's eye'' universal decoder. Every decoder we train is built for a specific task context $C$. For instance, ``these ten labels are the MNIST digits'' or ``the target is the distance of a pixel from image center.'' Because $C$ is baked into the trained decoder, the quantity we can bound in experiments is

\begin{equation}
H(I|R,C)
\label{eq:conditional-ambiguity}
\end{equation}

the entropy that remains given both the relational structure encoded in $R$ and the contextual constraint that interpretations must come from the known label set defined by $C$. By translating decoding performance into an upper bound on $H(I|R,C)$, we obtain a lower bound on how much ambiguity the training process has eliminated within that context.

This shifts our theory-experiment link from $H(I|R)$ to $H(I|R,C)$, but preserves the central idea: less ambiguous representations are those that admit fewer alternative interpretations even when the task is specified.

\subsection{The Ambiguity-Reduction Score (ARS)}

To quantify ambiguity reduction, we define the Ambiguity-Reduction Score:

\begin{equation}
\mathrm{ARS} = 1 - \frac{H(I|R,C)}{H_{\max}}
\label{eq:ars}
\end{equation}

where $H(I|R,C)$ is the conditional entropy of interpretations $I$ given a representation $R$ under task context $C$, and $H_{\max}$ is the entropy of a completely ambiguous representation:
\begin{itemize}
    \item For classification: $H_{\max} = \log_2 K$ where $K$ is the number of classes
    \item For regression: $H_{\max} = h(Y)$ where $h(Y)$ is the differential entropy of the target variable
\end{itemize}

The ARS ranges from 0 to 1:
\begin{itemize}
    \item $\mathrm{ARS} \approx 0$ indicates maximally ambiguous representations
    \item $\mathrm{ARS} \approx 1$ indicates fully unambiguous representations
\end{itemize}

\subsection{Deriving ARS from Classification Accuracy}

For classification tasks, Fano's inequality provides a lower bound on conditional entropy from top-1 accuracy $A$:

\begin{equation}
H(I|R,C) \leq h_b(1-A) + (1-A)\log_2(K-1)
\label{eq:fano}
\end{equation}

where $h_b(p) = -p\log_2(p) - (1-p)\log_2(1-p)$ is the binary entropy function. This yields:

\begin{equation}
\boxed{
\mathrm{ARS} \geq 1 - \frac{h_b(1-A) + (1-A)\log_2(K-1)}{\log_2 K}
}
\label{eq:ars-classification}
\end{equation}

Note that since this bound relies only on top-1 accuracy (treating every mistake as if any of the other $K-1$ classes could be correct), it overestimates residual ambiguity. Thus, the reported ARS values are \textit{conservative lower bounds} on the true ambiguity reduction.

\subsection{Deriving ARS from Regression Performance}

For regression tasks, assuming Gaussian residuals and standardizing the target so $\mathrm{Var}(Y) = 1$, the conditional entropy is bounded by:

\begin{equation}
H(Y|R,C) \leq \frac{1}{2}\log_2(2\pi e \cdot (1-R^2))
\label{eq:regression-entropy}
\end{equation}

where $R^2$ is the coefficient of determination. This leads to:

\begin{equation}
\boxed{
\mathrm{ARS} \geq \frac{\log_2[1/(1-R^2)]}{\log_2(2\pi e)} \approx \frac{\log_2[1/(1-R^2)]}{4.094}
}
\label{eq:ars-regression}
\end{equation}

\subsection{Results: Experiment 1 (MNIST Class Identity)}

For Experiment 1, we use the Gram matrix matching accuracies, which provide stronger and more stable bounds than the self-attention decoder. Table~\ref{tab:ars-exp1} shows the results.

\begin{table}[h]
\centering
\caption{Ambiguity-Reduction Scores for MNIST class identity decoding (Experiment 1). ARS calculated using Equation~\ref{eq:ars-classification} with $K=10$ classes.}
\label{tab:ars-exp1}
\begin{tabular}{lcc}
\toprule
\textbf{Training Paradigm} & \textbf{Accuracy} & \textbf{ARS (lower bound)} \\
\midrule
Dropout           & 1.000 & 1.000 \\
No Dropout        & 0.383 & 0.122 \\
Untrained         & 0.120 & 0.001 \\
\bottomrule
\end{tabular}
\end{table}

The results reveal striking differences in representational ambiguity:

\begin{itemize}
    \item \textbf{Dropout networks:} Achieve $\mathrm{ARS} = 1.000$, indicating \textit{perfect} unambiguity. The representation fully determines class identity with zero remaining entropy. This represents a complete solution to the ambiguity problem for categorical representations in this domain.

    \item \textbf{Standard backpropagation:} Achieve $\mathrm{ARS} = 0.122$, indicating substantial remaining ambiguity despite successful task learning. While these networks classify MNIST with $\approx 97\%$ accuracy, their internal representations remain largely ambiguous from a relational perspective.

    \item \textbf{Untrained networks:} Achieve $\mathrm{ARS} = 0.001$, confirming near-total ambiguity in random representations.
\end{itemize}

These results demonstrate that task performance alone does not guarantee unambiguous representations. Dropout networks and standard networks achieve identical classification accuracy, yet differ dramatically in representational ambiguity (ARS = 1.000 vs. 0.122). This dissociation suggests that unambiguity is a distinct property from task performance—one that may be particularly relevant for consciousness.

\subsection{Results: Experiment 2 (Spatial Position)}

For Experiment 2, we apply Equation~\ref{eq:ars-regression} to the $R^2$ scores for distance-from-center prediction. Table~\ref{tab:ars-exp2} shows the results.

\begin{table}[h]
\centering
\caption{Ambiguity-Reduction Scores for spatial position decoding (Experiment 2). ARS calculated using Equation~\ref{eq:ars-regression}.}
\label{tab:ars-exp2}
\begin{tabular}{lcc}
\toprule
\textbf{Training Paradigm} & \textbf{$R^2$} & \textbf{ARS (lower bound)} \\
\midrule
No Dropout        & 0.844 & 0.654 \\
Dropout           & 0.695 & 0.419 \\
Untrained         & $-0.008$ & 0.000 \\
\bottomrule
\end{tabular}
\end{table}

The spatial decoding results show a different pattern:

\begin{itemize}
    \item \textbf{Standard backpropagation:} Achieve $\mathrm{ARS} = 0.654$, indicating that spatial information is substantially less ambiguous than categorical information was in standard networks (0.122). This suggests spatial structure may be more naturally encoded in weight patterns.

    \item \textbf{Dropout networks:} Achieve $\mathrm{ARS} = 0.419$, surprisingly \textit{lower} than standard networks. This contrasts with the categorical case and suggests that dropout's effect on representation depends on the type of information being encoded.

    \item \textbf{Untrained networks:} Achieve $\mathrm{ARS} = 0.000$, again confirming that random weights encode no meaningful structure.
\end{itemize}

\subsection{Interpretation: Context-Conditioned Ambiguity}

Our experimental measurements yield $H(I|R,C)$ rather than the theoretically ideal $H(I|R)$. Does this conditional formulation undermine the framework? We argue that it does not, for several reasons:

\begin{enumerate}
    \item \textbf{Context is partially encoded in representations:} The dataset classification results (Section~\ref{sec:exp1}) demonstrate that key components of $C$ can be inferred from $R$ itself. We achieve 99.8\% accuracy distinguishing MNIST from Fashion-MNIST networks based solely on relational structure, showing that task domain—a major component of $C$—is actually encoded in $R$. This suggests that the distinction between $I$ (content to be decoded) and $C$ (context) is somewhat artificial, arising only because our decoder operates in a limited domain.

    \item \textbf{Toward universal decoders:} A universal decoder trained on relational structures across large, multi-modal neural networks could potentially eliminate the need for explicit context conditioning. The cross-architecture transfer results suggest this is feasible—relational structure is architecture-invariant and reflects properties of learned distributions rather than network specifics.

    \item \textbf{Biological context:} For biological consciousness, the relevant context $C$ is presumably quite broad—roughly ``our part of the universe and our sensory modalities.'' This is not an arbitrarily narrow context but rather the natural domain of organismal experience. Our results suggest that within appropriately broad contexts, representations can achieve the required unambiguity.
\end{enumerate}

\subsection{Practical Implications}

The ARS metric provides several practical benefits:

\begin{enumerate}
    \item \textbf{Quantifying representational quality:} ARS offers a principled measure of representation quality beyond task performance, potentially useful for comparing neural network architectures, training procedures, and regularization schemes.

    \item \textbf{Identifying consciousness-relevant representations:} If unambiguous representations are necessary for consciousness, ARS provides a quantitative tool for identifying which neural representations might support conscious content.

    \item \textbf{Guiding architecture search:} Training procedures that maximize ARS (like dropout for categorical representations) might be preferable for applications requiring interpretable, unambiguous representations.

    \item \textbf{Measuring alignment:} The connection between ARS and representational alignment (see Section~\ref{sec:discussion}) suggests that ARS might serve as a metric for comparing representations across different systems.
\end{enumerate}

\subsection{Summary}

Our quantitative analysis reveals:

\begin{itemize}
    \item Neural networks can achieve \textbf{perfect unambiguity} (ARS = 1.0) for categorical representations when trained with appropriate regularization.
    \item Spatial representations achieve \textbf{substantial unambiguity} (ARS $\approx$ 0.65) even without specialized training procedures.
    \item Task performance and representational ambiguity are \textbf{partially dissociable}—networks can classify perfectly while maintaining ambiguous or unambiguous internal representations.
    \item Different types of information (categorical vs. spatial) exhibit \textbf{different ambiguity profiles}, suggesting multiple mechanisms for unambiguous representation.
\end{itemize}

These findings provide quantitative support for the theoretical claim that neural networks can develop the unambiguous relational representations required by the intentionality constraint on conscious experience.

\section{Discussion}
\label{sec:discussion}

\subsection{Summary of Main Findings}

This work makes both theoretical and empirical contributions to understanding how neural systems can represent information unambiguously—a necessary condition for consciousness according to our intentionality constraint. Theoretically, we formalized the requirement that conscious representations must be unambiguous in terms of conditional entropy $H(I|R)$, and we proposed that relational structures can satisfy this requirement by intrinsically encoding both content and its interpretation. Empirically, we demonstrated that neural networks trained on image classification naturally develop such unambiguous relational representations, achieving perfect decoding accuracy (ARS = 1.0) for categorical information and substantial accuracy (ARS = 0.65) for spatial information.

\subsection{Implications for Consciousness Research}

\subsubsection{The Intentionality Constraint as a Filter for NCCs}

Our theoretical framework provides a principled filter for evaluating proposed neural correlates of consciousness. Any candidate NCC must explain not only \textit{that} a conscious experience occurs, but \textit{why} it has specific content rather than other possible contents. This intentionality constraint (Definition~\ref{def:intentionality}) rules out representations that require arbitrary decoding schemes, such as:

\begin{itemize}
    \item Pure population codes where neuron identity is arbitrary
    \item Representations requiring learned read-out mechanisms that could be applied differently
    \item Indexical representations (like bit strings) without relational structure
\end{itemize}

Instead, the constraint favors representations where meaning emerges from intrinsic structure—particularly relational structures where each element's meaning is determined by its relationships to other elements.

\subsubsection{Connection to Integrated Information Theory}

Our framework resonates with Integrated Information Theory (IIT)~\cite{haun2019space,kleiner2024mathematical}, which also emphasizes the importance of intrinsic structure and relations for consciousness. However, our approach differs in emphasis:

\begin{itemize}
    \item \textbf{IIT focus:} Integration and irreducibility of experience; the maximally irreducible conceptual structure (MICS) as the substrate of consciousness
    \item \textbf{Our focus:} Unambiguity and determinacy of content; relational structure as the means to achieve unambiguous representation
\end{itemize}

These perspectives are complementary. IIT's integration requirements might explain \textit{why} certain structures are conscious (they cannot be reduced to independent parts), while our unambiguity requirement explains \textit{what} the conscious content is about (the relational structure unambiguously specifies content).

Notably, we distinguish between two types of structuralism (Section~\ref{sec:theory}): (1) content determined by relations to all \textit{possible} experiences, and (2) content determined by relations instantiated \textit{in the current moment}. We argue that type (2) is necessary for explaining the determinacy of current experience, though these two types may ultimately be related.

\subsubsection{Functional vs. Structural Connectivity}

We acknowledged that our experiments examine structural connectivity (synaptic weights) rather than functional connectivity (activity correlations), which may be more appropriate for grounding conscious representations. However, structural connectivity should constrain and enable functional connectivity patterns. The fact that we can decode representational content from structural connectivity suggests that the learned weight patterns reflect the functional relationships that would arise during processing.

Future work should explicitly test whether functional connectivity (measured through activity correlations during stimulus processing) provides even more unambiguous representations than structural connectivity. We predict that functional connectivity will show higher ARS values, particularly for representations more directly related to current sensory input.

\subsection{Connection to the Platonic Representation Hypothesis}

Recent work by Huh et al.~\cite{huh2024platonic} proposed the Platonic Representation Hypothesis: that neural networks trained on different tasks and modalities converge toward a shared representation space reflecting the statistical structure of reality. They used mutual k-nearest neighbor (k-NN) kernel similarity to measure representational alignment across models.

Our findings complement and extend this work:

\begin{itemize}
    \item \textbf{Theoretical connection:} The Platonic Representation Hypothesis implicitly assumes that reality has a determinate structure that can be captured in representations. Our framework makes this explicit: unambiguous representations mirror the relational structure of the world, and this mirroring is what allows multiple networks to converge toward similar representations.

    \item \textbf{Empirical connection:} Figure~\ref{fig:knn-similarity} shows that k-NN kernel similarity correlates with decoder accuracy (and thus inversely with ambiguity) in our MNIST networks.

    \item \textbf{Ambiguity as a mechanism:} We propose that representational convergence occurs \textit{because} training reduces ambiguity. Networks that learn unambiguous relational structures necessarily align with each other and with the structure of reality.
\end{itemize}

\begin{figure}[h]
\centering
\includegraphics[width=0.7\textwidth]{../figures/knn-kernel-similarity-vs-decoder-accuracy.png}
\caption{k-NN kernel similarity vs. decoder accuracy. Mutual k-NN kernel similarity (computed by comparing output weights across networks trained with different seeds) correlates with decoding accuracy. This suggests that higher kernel similarity indicates lower representational ambiguity, potentially extending to large-scale multimodal models.}
\label{fig:knn-similarity}
\end{figure}

This connection suggests that if the pattern holds for larger, multimodal models, then higher representational alignment (as measured by k-NN similarity) should correlate with lower ambiguity. This would provide a practical tool for identifying which representations in large models achieve the unambiguity required for consciousness-relevant representations.

\subsection{Architectural and Training Considerations}

\subsubsection{The Role of Dropout}

Our most striking finding is that dropout regularization produces dramatically more unambiguous representations for categorical information (ARS = 1.0 vs. 0.12), despite identical task performance. This effect likely arises because dropout forces distributed representations by preventing co-adaptation of neurons~\cite{baldi2013understanding}.

For categorical information, this distribution creates distinctive relational geometries: neurons representing similar categories develop similar weight patterns because they must rely on overlapping features. For spatial information, however, dropout slightly reduces unambiguity (ARS = 0.42 vs. 0.65). This suggests that spatial structure benefits from more localized representations where individual weights can encode spatial relationships more directly.

These findings have practical implications:
\begin{itemize}
    \item For applications requiring interpretable categorical representations, dropout should be strongly favored
    \item For spatial or continuous representations, standard training may suffice
    \item Different regularization schemes might optimize different aspects of representational unambiguity
\end{itemize}

\subsubsection{Architecture Invariance}

The cross-architecture transfer results (Figures~\ref{fig:arch-transfer-decoder} and~\ref{fig:arch-transfer-gram}) demonstrate that relational structure is largely invariant to architectural details. This has important theoretical implications:

\begin{itemize}
    \item Relational representations reflect properties of the \textit{learned distribution} rather than architectural specifics
    \item The same informational content can be encoded in networks of different sizes and shapes
    \item This suggests that consciousness might not depend on specific architectural features, but rather on achieving appropriate relational structure—regardless of how that structure is implemented
\end{itemize}

This architecture invariance parallels the substrate independence often discussed in philosophy of mind: if consciousness depends on functional/relational organization rather than specific physical implementation, then different physical substrates could support the same conscious content provided they instantiate the same relational structure.

\subsection{Limitations and Future Directions}

\subsubsection{Limitations of Current Work}

Several limitations constrain our conclusions:

\begin{enumerate}
    \item \textbf{Simple domains:} MNIST and Fashion-MNIST are relatively simple, structured datasets. Whether similar unambiguity emerges in more complex, naturalistic domains remains to be tested.

    \item \textbf{Structural vs. functional connectivity:} Our analysis of synaptic weights rather than activity patterns may not capture the most consciousness-relevant aspects of neural representation.

    \item \textbf{Context conditioning:} Our measurements yield $H(I|R,C)$ rather than $H(I|R)$, requiring context about task domain. While we argue this is not problematic (Section~\ref{sec:ambiguity}), a truly universal decoder would be preferable.

    \item \textbf{Feedforward networks:} We examined only feedforward architectures. Recurrent networks, which are more biologically plausible and can implement dynamic processes, might show different patterns of representational ambiguity.

    \item \textbf{Single time point analysis:} Consciousness is a dynamic process unfolding over time. Our analysis of static connectivity patterns cannot capture temporal aspects of representation.
\end{enumerate}

\subsubsection{Future Research Directions}

Several promising directions emerge from this work:

\paragraph{Extending to complex domains:}
Applying these methods to natural image datasets (ImageNet, video data) and multimodal representations (vision-language models) would test whether unambiguous relational representations scale to realistic complexity. We predict that larger models trained on richer data will show even higher ARS values as they capture more of the relational structure of reality.

\paragraph{Functional connectivity analysis:}
Measuring representational ambiguity from activity patterns during stimulus processing would better match theoretical requirements for consciousness. This could involve computing correlations between neuron activations during natural stimulus presentation and testing whether these functional relationships provide more unambiguous representations than static weights.

\paragraph{Temporal dynamics:}
Extending the framework to recurrent networks and analyzing how representations evolve over time during stimulus processing would address the dynamic nature of consciousness. Time-resolved ARS measurements could track how ambiguity changes as information propagates through the network.

\paragraph{Biological neural networks:}
Applying relational decoding to neurophysiological data (e.g., multi-electrode recordings) would test whether biological neural networks also develop unambiguous relational representations. This might require new methods for estimating functional connectivity from spike trains.

\paragraph{Universal decoders:}
Training decoders on diverse tasks and modalities could reduce context dependence, moving from $H(I|R,C)$ toward $H(I|R)$. This might reveal whether truly context-free unambiguous representations are achievable.

\paragraph{Comparing theories:}
Using ARS as a common currency to evaluate different theories of consciousness. Which neural representations achieve the highest ARS: those maximizing integration (IIT), global availability (Global Workspace Theory), or predictive accuracy (Predictive Processing)?

\subsection{Philosophical Implications}

\subsubsection{The Hard Problem and Unambiguity}

Our work does not solve the hard problem of consciousness—explaining why and how physical processes give rise to subjective experience. However, it addresses an important prerequisite: the intentionality constraint. Before we can explain why certain physical processes are conscious, we must explain why they have \textit{specific} content. Our framework shows that relational structures can satisfy this requirement by achieving unambiguous representation.

This suggests that the hard problem might decompose into sub-problems:
\begin{enumerate}
    \item How do physical systems achieve unambiguous representations? (Addressed by our work)
    \item Why do unambiguous representations (or certain types of them) give rise to subjective experience? (Remains open)
    \item What additional properties (integration, global availability, etc.) are required beyond unambiguity? (Requires further investigation)
\end{enumerate}

\subsubsection{Representationalism Revisited}

Our framework supports a sophisticated form of representationalism: conscious experiences are representations, but not all representations are conscious. Specifically, only \textit{unambiguous} relational representations that intrinsically specify their content could be conscious. This explains why:

\begin{itemize}
    \item Random neural activity is not conscious (ARS $\approx$ 0)
    \item Simple feedforward processing might not be conscious even when task-relevant (ARS can be low despite good performance)
    \item Highly integrated, relational representations are candidates for consciousness (ARS $\approx$ 1)
\end{itemize}

This suggests that consciousness requires not just representation, but a specific \textit{kind} of representation—one that achieves the unambiguity required for determinate content.

\subsection{Implications Beyond Consciousness}

While motivated by consciousness research, our framework has broader implications:

\paragraph{Interpretability:}
ARS provides a quantitative measure of how unambiguous neural network representations are, potentially useful for evaluating interpretability. More unambiguous representations should be more interpretable to external observers.

\paragraph{Robustness:}
Unambiguous representations might be more robust to perturbations, since their meaning is intrinsically determined rather than dependent on specific decoding schemes.

\paragraph{Transfer learning:}
Representations with low ambiguity might transfer better across tasks, since they capture fundamental structure rather than task-specific patterns.

\paragraph{AI safety:}
Understanding what neural networks truly represent (reducing ambiguity in our interpretation of their representations) is crucial for ensuring they behave as intended.

\subsection{Conclusion of Discussion}

Our theoretical and empirical investigation demonstrates that:
\begin{enumerate}
    \item The intentionality constraint provides a principled requirement for consciousness-relevant representations
    \item Relational structures can satisfy this constraint by achieving unambiguous representation
    \item Neural networks naturally develop such structures through learning
    \item The degree of unambiguity can be quantified using the ARS metric
    \item Different training procedures and information types yield different ambiguity profiles
\end{enumerate}

These findings bridge theoretical philosophy of mind, computational neuroscience, and machine learning, providing both conceptual clarity and practical tools for investigating the neural basis of conscious content.

\section{Conclusion}
\label{sec:conclusion}

The neural basis of consciousness must explain not only \textit{that} certain brain states correspond to conscious experiences, but \textit{why} they correspond to specific experiences with determinate content. This paper addressed this intentionality constraint by formalizing the requirement that conscious representations be unambiguous—they must intrinsically specify what they represent without relying on arbitrary decoding schemes.

We proposed that relational structures, where meaning emerges from patterns of relationships between elements, can achieve the required unambiguity. Drawing on information theory, we formalized representational ambiguity as $H(I|R)$, the conditional entropy of possible interpretations given a representation. Through experiments with artificial neural networks trained on image classification, we demonstrated that:

\begin{itemize}
    \item Networks naturally develop unambiguous relational representations through learning, achieving perfect decoding accuracy (ARS = 1.0) for categorical information when trained with dropout regularization
    \item Both categorical (digit class) and spatial (pixel position) information can be decoded purely from relational structure, with neuron identities deliberately obscured
    \item The degree of unambiguity is largely independent of task performance—networks can classify perfectly while maintaining ambiguous or unambiguous internal representations
    \item Relational structure is architecture-invariant, suggesting it reflects properties of learned distributions rather than network specifics
    \item Different types of information (categorical vs. spatial) exhibit different ambiguity profiles, indicating multiple mechanisms for unambiguous representation
\end{itemize}

We introduced the Ambiguity-Reduction Score (ARS), which quantifies how much training reduces representational ambiguity by bounding conditional entropy from decoder performance. This metric provides a practical tool for evaluating representational quality beyond task accuracy, with potential applications in interpretability, robustness, and consciousness research.

Our findings connect to broader questions in philosophy of mind, neuroscience, and artificial intelligence. They support sophisticated representationalism: conscious experiences may be representations, but only certain kinds—unambiguous relational representations that intrinsically specify content. They complement Integrated Information Theory by focusing on determinacy and unambiguity rather than integration. They align with the Platonic Representation Hypothesis by suggesting that representational convergence occurs because training reduces ambiguity toward the intrinsic structure of reality.

While this work does not solve the hard problem of consciousness—explaining why physical processes give rise to subjective experience—it addresses an important prerequisite by showing how physical systems can achieve representations with determinate content. The intentionality constraint provides a principled filter for evaluating proposed neural correlates: any candidate NCC must explain not just correlation but content specificity, ruling out representations that require arbitrary interpretation.

Several important questions remain open. Does functional connectivity (activity correlations) provide even more unambiguous representations than the structural connectivity we examined? Do biological neural networks develop similar relational structures? Can we build truly universal decoders that eliminate context dependence, measuring $H(I|R)$ rather than $H(I|R,C)$? What additional properties beyond unambiguity—integration, global availability, recurrence—are required for consciousness?

Nevertheless, our results demonstrate that the problem of representational ambiguity—long considered a philosophical puzzle—can be approached through rigorous experimental investigation. Neural networks provide tractable model systems for studying how physical systems develop unambiguous representations. The methods developed here—relational preprocessing, geometric structure matching, and the ARS metric—offer practical tools for quantifying and comparing representations across systems.

Ultimately, understanding consciousness requires understanding representation. By formalizing and measuring representational ambiguity, we take a step toward explaining not just which neural activity correlates with conscious experience, but why it does—why certain patterns of activity are \textit{about} seeing apples rather than oranges, experiencing red rather than blue, or feeling pain rather than pleasure. The unambiguity of relational representations provides a foundation for determinate content, bringing us closer to bridging the explanatory gap between physical processes and phenomenal experience.


% References
\bibliographystyle{plain}
\bibliography{references}

\end{document}
