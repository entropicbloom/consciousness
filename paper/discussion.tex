\subsection{Implications for Consciousness Research}

Our experimental results demonstrate that neural networks can encode information through relational structure in a way that allows unambiguous decoding of representational content. For dropout-trained networks, we achieved perfect (100\%) accuracy in identifying output neuron class identity from connectivity patterns alone. This operationalizes the theoretical claim that conscious representations must be unambiguous: the neural state itself specifies what it represents.

A potential objection is that our decoding methods constitute external decoders. However, unlike arbitrary conventions (e.g., JPEG), our decoders \emph{discover} structure rather than impose it. Both geometric matching and learned decoders infer consistent relational patterns from network instances and generalize across architectures. This demonstrates that the decoder is not arbitrary but extracts structure implicitly present in the networks—the content is structurally determined, not conventionally assigned.

We should clarify what "unambiguous" means in this context, as it refers to informational determinacy rather than phenomenal clarity. A neural state has unambiguous representational content when that content is fixed and determinate. Consider peripheral vision: the phenomenal character itself lacks detail, yet the neural state has determinate content representing "indeterminate peripheral presence." Our framework addresses semantic determinacy (what the representation is about) rather than phenomenal determinacy (whether the experience feels clear).

How does this connect to consciousness? According to representationalist theories \citep{lycan2019representational,pennartz2018consciousness}, conscious experience is determined by intentional contents: what the brain represents. Our work adds a constraint on the \emph{manner} in which these contents must be represented: they must be represented unambiguously. This constraint resonates with Chalmers' notion of \emph{impure representationalism} \citep{chalmers2004representational}, where not all representations give rise to consciousness, only those satisfying particular conditions.

Importantly, while we have framed this work within narrow representationalism, the core constraint emerges more generally from consciousness being intrinsic. The same intuition can be expressed using IIT terminology \citep{tononi2016integrated}: consciousness is both \emph{informative} and \emph{intrinsic}, implying that the neural substrate must unambiguously specify conscious contents. We acknowledge that narrow representationalism is controversial, so our framework should be understood conditionally: \emph{if} conscious experience is determined by brain-internal states alone, \emph{then} those states must encode content unambiguously.

\subsection{Training Paradigm Effects}

A striking finding is that dropout training produces dramatically more unambiguous representations than standard backpropagation for output neurons (100\% vs.\ 38\% geometric matching accuracy), despite nearly identical task performance. This dissociation suggests that representational ambiguity is largely orthogonal to classification accuracy: a network can perform well while encoding information ambiguously.

Intuitively, dropout encourages distributed representations by preventing reliance on single-neuron pathways \citep{baldi2013understanding}. This forces output neurons representing similar classes to have similar input weight patterns, as they must share overlapping features. The result is a more distinctive relational geometry where each class occupies a unique, well-separated position.

We do not claim that dropout-trained networks are more conscious than standard networks, or that consciousness requires dropout specifically. Rather, dropout serves as an existence proof that training regimes can produce varying levels of representational ambiguity while maintaining task performance. The specific mechanism is less important than the principle: ambiguity is a dimension of neural representation that can be optimized independently of behavioral performance, at least within the performance regime we tested. Whether this independence holds more generally—particularly at the limits of task complexity or performance—remains an open question.

Interestingly, the pattern reverses for input neurons: standard backpropagation yields higher spatial position decoding accuracy than dropout ($R^2 = 0.844$ vs.\ $R^2 = 0.695$). We do not speculate extensively on the mechanism, but note that different layers may be subject to different optimization dynamics. The key point is that training paradigm substantially affects representational ambiguity in ways that are not captured by task performance metrics.

\subsection{Broader Connections}

Our methodology relates to representational similarity analysis (RSA) in computational neuroscience \citep{kriegeskorte2008representational,yamins2016using} and recent work on representation alignment across networks \citep{huh2024platonic}. We extend RSA by comparing similarity structures across network instances rather than stimuli, and by using this comparison to decode specific content rather than merely assess alignment.

\subsection{Limitations and Future Directions}

Our experiments have several important limitations. First, we focus exclusively on feedforward networks trained with supervised learning on MNIST, a simple, well-structured task. Whether similar unambiguous encoding emerges in more complex scenarios remains open: unsupervised or self-supervised learning, naturalistic datasets with richer statistical structure, recurrent networks with temporal dynamics, or deeper hierarchical architectures.

Second, we measure $H(I|R,C)$ rather than $H(I|R)$, where $C$ represents context (neural network domain, dataset, architecture, decoding target). Does this fundamentally limit our conclusions? Several lines of evidence suggest the limitations are practical rather than fundamental. Dataset identity itself can be inferred from relational structure (99.8\% accuracy), suggesting the $I$/$C$ distinction is somewhat artificial. Cross-architecture transfer suggests that architecture is not a critical component of context. A universal decoder trained on diverse, multi-modal neural networks could potentially approach true $H(I|R)$ by learning to infer context from structure itself.

Third, our experiments focused on structural connectivity (synaptic weights) rather than functional connectivity (activity correlations). While functional connectivity may be more relevant for consciousness due to its dynamic, state-dependent nature \citep{pennartz2009identification}, structural connectivity provides a stable proxy in these simple feedforward networks where structural and functional patterns are tightly coupled. For trained networks in a stable state, structural connectivity captures the learned representational geometry. Nevertheless, future work should analyze functional connectivity patterns during actual stimulus processing, particularly in biological neural recordings.

Fourth, our analysis examines single layers in isolation. A complete understanding would require examining how relational structures compose across layers and whether hierarchy introduces or reduces ambiguity. Our relational complexity scaling results reveal an important principle: more complex relational structures enable more unambiguous representations. Performance relative to random chance increases systematically from 1.0x (2 neurons) to 7.36x (10 neurons), demonstrating that richer relational geometries provide more constraints on possible interpretations, thereby reducing representational ambiguity.

Finally, the ultimate test would be applying this framework to biological neural data. Can we measure representational ambiguity in visual cortex recordings, and does it correlate with conscious perception? Do neural populations representing consciously perceived stimuli exhibit lower ambiguity than those processing unconscious information? These questions could provide empirical bridges between computational principles and phenomenal experience.
